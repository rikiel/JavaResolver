
\chapter{Introduction}



\section{Motivation}

Data lineage, or provenance, describes where data came from,
how it was derived and where the results are stored.

Lineage can be useful in many domains.
Molecular biology databases, which mostly store copied data,
can use lineage to verify the copied data by tracking the original
sources as was demonstrated by \citet{DataLineageInBioInformatics}.
Probabilistic databases can exploit lineage for confidence
computation as was stated by \citet{DataLineageInProbabilisticDatabases}.
Data lineage is also important in BI when achieving full regulatory compliance
and improving data governance.

In all these areas, Java applications are often used to work
with such a data.
Until now, there exist no tool for creating data lineage of
Java applications and the fact, that there exist many frameworks
to simplify and abstract access to that data,
makes the task of creating data lineage even harder.

When talking about databases, useful information can be the 
name of a database that application access or SQL statement that application execute.
Application can also use external files as data storage, where their names
are very valuable information for data lineage.
Data can be also accessed through network using messaging system
where server name or message identification can be also useful.

Resulted data lineage can be viewed as a graph, where nodes
contains sources and sinks of data and directed edges
between them are corresponding data flows.




\section{MANTA Flow}

\citet{MantaFlow} is a tool that helps enterprises to get end-to-end
data lineage including custom SQL code. That allows customers to fulfill
compliance regulations or improve data governance.

It extracts and analyses metadata from report definitions,
custom SQL code, and ETL workflows, to create data flows
which span multiple systems and a range of technologies.
Lineages are analyzed based on actual code.
All entities detected by MANTA Flow, such as columns or procedures,
are visualized to help users utilize this information.

Systems, for which MANTA Flow is used to analyse their data lineage,
often use Java applications to work with data. Adding support
for data lineage of such programs would be valuable,
as presented data lineage result can cover whole system,
not only its database parts.




\section{Goals}

Many Java applications use other frameworks to access data
in databases, files or in network storages.

Up to now, there exists the Symbolic Analysis library
that can create data lineage of an analysed application
when only Java IO and JDBC APIs are used.

In our work, we \textbf{propose architecture changes} in Symbolic Analysis library
to be easily extensible by plugins, that can add support for new frameworks
to identify its data sources and sinks.

We also \textbf{implement plugins for few selected frameworks}
(\citet{MyBatis}, \citet{SpringJDBC} and \citet{Kafka}).
Each framework has different approach for accessing the data.
Thereby we want to demonstrate that such plugins can be used to
add support for data flow analysis of other Java frameworks.
The feature to easily extend the library to add support for a new frameworks
is very important, as new and new frameworks are being developed today.




\section{Structure of the Work}

In chapter \ref{chapter:frameworks} we give examples of popular Java frameworks
for data processing.

In chapter \ref{chapter:analysis} we introduce libraries that are used
for static analysis to create data lineage.

In chapter \ref{chapter:implementation} we present design of the library for
data lineage of frameworks.

Chapter \ref{chapter:program} contains user documentation for the library.

Chapter \ref{chapter:results} present our results for selected frameworks,
constrains and future work.



