
\chapter{Introduction}

For modern data processing applications, an important aspect
is a data lineage.
Data lineage, or provenance, describes where data came from,
how it was derived and where the results are stored.

Lineage can be useful in many domains.
Molecular biology databases, which mostly store copied data,
can use lineage to verify the copied data by tracking the original
sources as was demonstrated by \citet{DataLineageInBioInformatics}.
Probabilistic databases can exploit lineage for confidence
computation as was stated by \citet{DataLineageInProbabilisticDatabases}.
The important usage of the data lineage is also in machine learning,
where for the purpose of a verification and validation can be important
to know data sources of used data sets.
Data lineage is also important in Business Intelligence (BI)
when achieving full regulatory compliance and improving data governance.

In all these areas, Java applications together with data manipulation frameworks
are often used to access data in databases, files or in network storages.
The tool that is able to identify the sources and sinks of a data
and create data lineage of an application can be very useful.

Up to now, we are aware of the only tool for creating data lineage of
Java applications - the Symbolic analysis library
presented in \citet{ParizekHybridAnalysis}.
The Symbolic analysis library performs a specific kind of static program analysis.
It can create data lineage of an analysed application when only
Java I/O and JDBC APIs are used in addition to the core
Java libraries (collections, etc.).



\section{MANTA Flow}

An example of a data lineage analysis tool is the \citet{MantaFlow}.
The MANTA Flow helps enterprises to get end-to-end
data lineage including custom SQL code. That allows customers to fulfill
compliance regulations or improve data governance.

It extracts and analyses metadata from report definitions,
custom SQL code, and extract-transform-load (ETL) workflows, to create data flow graphs
which span multiple systems and a range of technologies.
Lineages are computed based on analysis of actual code.
All entities detected by MANTA Flow, such as database tables, columns or procedures,
are visualized to help users utilize this information.

Systems, for which MANTA Flow is used to analyse their data lineage,
often use Java applications and data processing frameworks to work with data.
MANTA Flow integrates the Symbolic analysis library to create data lineage of
Java applications to cover whole system, not only its database parts.
However, the support for computing data lineage of complex data processing frameworks
is missing.




\section{Goals}

In our work, we \textit{propose architecture changes} to Symbolic analysis library
to be easily extensible by plugins that can add support for new data processing frameworks
to identify its data sources and sinks.

We also \textit{implement Symbolic analysis library plugins} for few selected frameworks
(MyBatis, Spring JDBC and Apache Kafka).
Each framework has different approach for accessing the data.
Thereby we want to demonstrate that such plugins can be used to
add support for data lineage analysis of other Java frameworks.
The feature to easily extend the Symbolic analysis library to add support for a new frameworks
is very important, as new frameworks are being developed today.




\section{Structure of the Work}

The rest of the thesis has the following structure.

In chapter \ref{chapter:frameworks} we give overview of popular Java frameworks
for data processing.

In chapter \ref{chapter:analysis} we introduce libraries that are used
for static analysis of Java programs to create data lineage.

In chapter \ref{chapter:implementation} we present design of the \ToolName tool for
data lineage of frameworks. We also describe some selected implementation details of the tool.

Chapter \ref{chapter:program} contains user documentation for the \ToolName tool.

Chapter \ref{chapter:results} present our results for selected frameworks,
limitations of our solution and plans for the future work.

