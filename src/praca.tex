%%% Hlavní soubor. Zde se definují základní parametry a odkazuje se na ostatní části. %%%

%% Verze pro jednostranný tisk:
% Okraje: levý 40mm, pravý 25mm, horní a dolní 25mm
% (ale pozor, LaTeX si sám přidává 1in)
\documentclass[12pt,a4paper]{report}
\setlength\textwidth{145mm}
\setlength\textheight{247mm}
\setlength\oddsidemargin{15mm}
\setlength\evensidemargin{15mm}
\setlength\topmargin{0mm}
\setlength\headsep{0mm}
\setlength\headheight{0mm}
% \openright zařídí, aby následující text začínal na pravé straně knihy
\let\openright=\clearpage


%%% Údaje o práci
\def\ToolName{Java Resolver }

% Název práce v jazyce práce (přesně podle zadání)
\def\NazevPrace{Analyzing Data Lineage in Database Frameworks}
% Jméno autora
\def\AutorPrace{Richard Eliáš}
% Rok odevzdání
\def\RokOdevzdani{2019}
% Název katedry nebo ústavu, kde byla práce oficiálně zadána
% (dle Organizační struktury MFF UK, případně plný název pracoviště mimo MFF)
\def\Katedra{Department of Distributed and Dependable Systems}
% Jedná se o katedru (department) nebo o ústav (institute)?
\def\TypPracoviste{Department}
% Vedoucí práce: Jméno a příjmení s~tituly
\def\Vedouci{RNDr. Pavel Parízek, Ph.D.}
% Pracoviště vedoucího (opět dle Organizační struktury MFF)
\def\KatedraVedouciho{Department of Distributed and Dependable Systems}
% Studijní program a obor
\def\StudijniProgram{Computer Science}
\def\StudijniObor{Artificial Inteligence}
% Nepovinné poděkování (vedoucímu práce, konzultantovi, tomu, kdo
% zapůjčil software, literaturu apod.)
\def\Podakovanie{%
  I would like to thank my brothers Marek and Erik,
  my mother and my father and all those who
  have always supported and moved me forward during my studies.

  I dedicate this thesis to my beautiful wife Anička,
  without the help of which I would not be able to
  master my studies.
}
% Abstrakt (doporučený rozsah cca 80-200 slov; nejedná se o zadání práce)
\def\Abstrakt{%
  Large information systems are typically implemented using frameworks and libraries.
  An important property of such systems is data lineage - the flow of data loaded
  from one system (e.g. database), through the program code, and back to another system.
  We implemented the \ToolName tool for data lineage analysis of Java programs based on
  the Symbolic analysis library for computing data lineage of
  simple Java applications. The library supports only JDBC and I/O APIs to identify
  the sources and sinks of data flow. We proposed some architecture changes
  to the library to make easily extensible by plugins that can add support for
  new data processing frameworks. We implemented such plugins for few
  frameworks with different approach for accessing the data,
  including Spring JDBC, MyBatis and Kafka.
  Our tests show that this approach works and can be usable in practice.
}
\def\AbstraktSK{%
  Informačné systémy často vo svojej implementácií využívajú už existujúce
  frameworky a knižnice. Dôležitou vlastnosťou takýchto systémov sú ich dátové toky.
  Dáta sú načítané zo zdrojového systému (napríklad databáze), pokračujú
  cez aplikačný kód a zapísané sú do ďalšieho, cieľového systému.
  Ako súčasť práce sme implementovali nástroj \ToolName schopný získavať
  dátové toky Java aplikácií. Program využíva existujúcu knižnicu
  Symbolic analysis library, ktorá je shopná počítať dátové toky jednoduchých
  Java aplikácií. Ako zdrojové a cieľové systémy však knižnica rozpoznáva
  iba JDBC a I/O API.
  V našom riešení sme navrhli zmenu architektúry knižnice, aby bola jednoducho
  rozšíriteľná pomocou pluginov, ktoré môžu pridať podporu pre nové frameworky.
  Tieto pluginy sme implementovali pre niekoľko frameworkov
  (Spring JDBC, MyBatis a Kafka). Naše testy ukazujú, že tento prístup
  môže byť v praxi využiteľný.
}
% 3 až 5 klíčových slov (doporučeno), každé uzavřeno ve složených závorkách
\def\KlicovaSlova{%
  data lineage, data flow visualization, static program analysis, Java frameworks
}


\usepackage{filecontents}
\begin{filecontents*}{\jobname.xmpdata}
  \Author{\AutorPrace}
  \Title{\NazevPrace}
  \Keywords{\KlicovaSlova}
  \Subject{\Abstrakt}
  \Publisher{Univerzita Karlova}
\end{filecontents*}

%% Vytváříme PDF/A-2u
\usepackage[a-2u]{pdfx}

\usepackage{lmodern}
\usepackage[T1]{fontenc}
\usepackage{textcomp}

%% Použité kódování znaků: obvykle latin2, cp1250 nebo utf8:
\usepackage[utf8]{inputenc}

%%% Další užitečné balíčky (jsou součástí běžných distribucí LaTeXu)
\usepackage{amsmath}        % rozšíření pro sazbu matematiky
\usepackage{amsfonts}       % matematické fonty
\usepackage{amsthm}         % sazba vět, definic apod.
\usepackage{bbding}         % balíček s nejrůznějšími symboly
			    % (čtverečky, hvězdičky, tužtičky, nůžtičky, ...)
\usepackage{bm}             % tučné symboly (příkaz \bm)
\usepackage{graphicx}       % vkládání obrázků
\usepackage{fancyvrb}       % vylepšené prostředí pro strojové písmo
\usepackage{indentfirst}    % zavede odsazení 1. odstavce kapitoly
\usepackage[numbers]{natbib}         % zajištuje možnost odkazovat na literaturu
			    % stylem AUTOR (ROK), resp. AUTOR [ČÍSLO]
\usepackage[nottoc]{tocbibind} % zajistí přidání seznamu literatury,
                            % obrázků a tabulek do obsahu
\usepackage{icomma}         % inteligetní čárka v matematickém módu
\usepackage{dcolumn}        % lepší zarovnání sloupců v tabulkách
\usepackage{booktabs}       % lepší vodorovné linky v tabulkách
\usepackage{paralist}       % lepší enumerate a itemize
\usepackage{float}

\usepackage{subcaption}
\usepackage[noend]{algpseudocode}
\usepackage{algorithm}
\usepackage{cases}
\usepackage{tikz}
\usepackage{amssymb}
\usepackage{xspace}

\usepackage[percent]{overpic}

\usepackage[export]{adjustbox}
\usepackage{listings}
\renewcommand\lstlistingname{Listing}
\renewcommand\lstlistlistingname{List of Listings}
\usepackage{color}


%% Balíček hyperref, kterým jdou vyrábět klikací odkazy v PDF,
%% ale hlavně ho používáme k uložení metadat do PDF (včetně obsahu).
\hypersetup{unicode}
\hypersetup{breaklinks=true}
\hypersetup{urlcolor=blue}

%% Definice různých užitečných maker (viz popis uvnitř souboru)
%%% Tento soubor obsahuje definice různých užitečných maker a prostředí %%%
%%% Další makra připisujte sem, ať nepřekáží v ostatních souborech.     %%%

%%% Drobné úpravy stylu

% Tato makra přesvědčují mírně ošklivým trikem LaTeX, aby hlavičky kapitol
% sázel příčetněji a nevynechával nad nimi spoustu místa. Směle ignorujte.
\makeatletter
\def\@makechapterhead#1{
  {\parindent \z@ \raggedright \normalfont
   \Huge\bfseries \thechapter. #1
   \par\nobreak
   \vskip 20\p@
}}
\def\@makeschapterhead#1{
  {\parindent \z@ \raggedright \normalfont
   \Huge\bfseries #1
   \par\nobreak
   \vskip 20\p@
}}
\makeatother

% Toto makro definuje kapitolu, která není očíslovaná, ale je uvedena v obsahu.
\def\chapwithtoc#1{
\chapter*{#1}
\addcontentsline{toc}{chapter}{#1}
}

% Trochu volnější nastavení dělení slov, než je default.
\lefthyphenmin=2
\righthyphenmin=2

% Zapne černé "slimáky" na koncích řádků, které přetekly, abychom si
% jich lépe všimli.
\overfullrule=1mm

%%% Makra pro definice, věty, tvrzení, příklady, ... (vyžaduje baliček amsthm)

\theoremstyle{plain}
\newtheorem{veta}{Veta}
\newtheorem{lemma}[veta]{Lemma}
\newtheorem{tvrz}[veta]{Tvdenie}

\theoremstyle{plain}
\newtheorem{definice}{Definícia}

\theoremstyle{remark}
\newtheorem*{dusl}{Dôsledok}
\newtheorem*{pozn}{Poznámka}
\newtheorem*{prikl}{Príklad}

%%% Prostředí pro důkazy

\newenvironment{dukaz}{
  \par\medskip\noindent
  \textit{Dôkaz}.
}{
\newline
\rightline{$\square$}  % nebo \SquareCastShadowBottomRight z balíčku bbding
}

%%% Prostředí pro sazbu kódu, případně vstupu/výstupu počítačových
%%% programů. (Vyžaduje balíček fancyvrb -- fancy verbatim.)

\DefineVerbatimEnvironment{code}{Verbatim}{fontsize=\small, frame=single}

%%% Prostor reálných, resp. přirozených čísel
\newcommand{\R}{\mathbb{R}}
\newcommand{\N}{\mathbb{N}}

%%% Užitečné operátory pro statistiku a pravděpodobnost
\DeclareMathOperator{\pr}{\textsf{P}}
\DeclareMathOperator{\E}{\textsf{E}\,}
\DeclareMathOperator{\var}{\textrm{var}}
\DeclareMathOperator{\sd}{\textrm{sd}}

%%% Příkaz pro transpozici vektoru/matice
\newcommand{\T}[1]{#1^\top}

%%% Vychytávky pro matematiku
\newcommand{\goto}{\rightarrow}
\newcommand{\gotop}{\stackrel{P}{\longrightarrow}}
\newcommand{\maon}[1]{o(n^{#1})}
\newcommand{\abs}[1]{\left|{#1}\right|}
\newcommand{\dint}{\int_0^\tau\!\!\int_0^\tau}
\newcommand{\isqr}[1]{\frac{1}{\sqrt{#1}}}

%%% Vychytávky pro tabulky
\newcommand{\pulrad}[1]{\raisebox{1.5ex}[0pt]{#1}}
\newcommand{\mc}[1]{\multicolumn{1}{c}{#1}}

\renewcommand{\O}[1]{\sloppy\mbox{\ensuremath{\mathcal{O}}(#1)}}

\newcommand{\scale}{0.6}
\newcommand{\subtree}[1]{\begin{tikzpicture}[
          on grid,
          font=\tiny,
          scale = \scale,
          node distance = 0.6 cm,
          sibling distance = 0.8 cm,
          text width = 0.2cm,
          level distance = 1.1 cm,
          colored/.style = {color = red},
          normal/.style = {black = red, draw, opacity = 100},
          invisible/.style={opacity = 0},
          cross/.style={path picture={ 
                \draw
                (path picture bounding box.south east) -- (path picture bounding box.north west) (path picture bounding box.south west) -- (path picture bounding box.north east);
            }},
          every node/.style = {scale = \scale, circle, draw, align = center},
          tree/.style = {draw = none, fill = none}]#1\end{tikzpicture}}

\definecolor{green}{rgb}{0,0.6,0}
\definecolor{gray}{rgb}{0.5,0.5,0.5}
\definecolor{yellow}{rgb}{0.73,0.71,0.16}
\definecolor{pink}{rgb}{0.58,0,0.82}
\definecolor{orange}{rgb}{0.8,0.47,0.19}


\lstdefinelanguage{Java}{
  morekeywords={
    return,try,catch,finally,
    static,default,final,volatile,new,
    null,void,int,long,boolean,short,double,float,
    interface,class,extends,implements,
    if,else,for,while,do,throw,throws,
    public,protected,private},
  numbers=left,
  numbersep=3mm,
  numberstyle=\tiny\color{gray},
  frame=tb,
  %aboveskip=3mm,
  %belowskip=3mm,
  showstringspaces=false,
  columns=flexible,
  basicstyle={\small\ttfamily},
  showlines=false,
  keywordstyle=\color{blue},
  commentstyle=\color{green},
  morecomment=[l]{//},
  stringstyle=\color{pink},
  morestring=[b]",
  morestring=[d]’,
  moredelim=[is][\textcolor{orange}]{@@}{@@},
  breaklines=true,
  breakatwhitespace=true,
  escapeinside={/*}{*/},
  captionpos=b,
}

\lstdefinelanguage{XML}{
  %language=HTML,
  numbers=left,
  numbersep=3mm,
  numberstyle=\tiny\color{gray},
  frame=tb,
  %aboveskip=3mm,
  %belowskip=3mm,
  showstringspaces=false,
  columns=flexible,
  basicstyle={\small\ttfamily},
  showlines=false,
  keywordstyle=\color{blue},
  commentstyle=\color{green},
  morecomment=[s]{<!--}{-->},
  stringstyle=\color{pink},
  moredelim=[is][\textcolor{blue}]{@}{@},
  moredelim=[is][\textcolor{orange}]{@@}{@@},
  morestring=[b]",
  morestring=[d]’,
  breaklines=true,
  breakatwhitespace=true,
  escapeinside={/*}{*/},
  captionpos=b,
}

\lstdefinelanguage{JavaSnippet}{
  morekeywords={
    return,try,catch,finally,
    static,default,final,volatile,new,
    null,void,int,long,boolean,short,double,float,
    interface,class,extends,implements,
    if,else,for,while,do,throw,throws,
    public,protected,private},
  numbers=none,
  numbersep=3mm,
  numberstyle=\tiny\color{gray},
  frame=none,
  aboveskip=5mm,
  belowskip=5mm,
  showstringspaces=false,
  columns=flexible,
  basicstyle={\small\ttfamily},
  showlines=false,
  keywordstyle=\color{blue},
  commentstyle=\color{green},
  morecomment=[l]{//},
  stringstyle=\color{pink},
  morestring=[b]",
  morestring=[d]’,
  moredelim=[is][\textcolor{orange}]{@@}{@@},
  breaklines=true,
  breakatwhitespace=true,
  escapeinside={/*}{*/},
  captionpos=b,
}

\lstdefinelanguage{shell}{
  numbers=none,
  numberstyle=\tiny\color{gray},
  frame=none,
  aboveskip=5mm,
  belowskip=5mm,
  showstringspaces=false,
  columns=flexible,
  basicstyle={\small\ttfamily},
  showlines=false,
  keywordstyle=\color{blue},
  commentstyle=\color{green},
  morecomment=[s]{<!--}{-->},
  stringstyle=\color{pink},
  moredelim=[is][\textcolor{blue}]{@}{@},
  moredelim=[is][\textcolor{orange}]{@@}{@@},
  morestring=[b]",
  morestring=[d]’,
  breaklines=true,
  breakatwhitespace=true,
  escapeinside={/*}{*/},
  captionpos=b,
}

% Prepinace pre figure:
% h=approximately here
% t=top of page
% b=bottom of page
% p=on special page
% H=precisely here
\newcommand{\InsertCode}[2]{\begin{figure}[#1]\input{#2}\end{figure}}

\newcommand{\Code}[1]{\texttt{#1}}

\newcommand{\ToolName}{Java Resolver }

\makeatletter
\AtBeginDocument{%
  \let\c@table\c@lstlisting
  \let\thetable\thelstlisting

  \let\c@figure\c@lstlisting
  \let\thefigure\thelstlisting
  \let\ftype@lstlisting\ftype@figure
}
\makeatother

\newcommand\TODO[1]{{\color{red}TODO #1}}
\newcommand{\uvodzovky}[1]{``#1''}



%% Titulní strana a různé povinné informační strany
\begin{document}
%%% Title page of the thesis and other mandatory pages

%%% Title page of the thesis

\pagestyle{empty}
\hypersetup{pageanchor=false}
\begin{center}

\centerline{\mbox{\includegraphics[width=166mm]{img/logo-en.pdf}}}

\vspace{-8mm}
\vfill

{\bf\Large MASTER THESIS}

\vfill

{\LARGE\AutorPrace}

\vspace{15mm}

{\LARGE\bfseries\NazevPrace}

\vfill

\Katedra

\vfill

\begin{tabular}{rl}
Supervisor of the master thesis: & \Vedouci \\
\noalign{\vspace{2mm}}
Study programme: & \StudijniProgram \\
\noalign{\vspace{2mm}}
Study branch: & \StudijniObor \\
\end{tabular}

\vfill

% Zde doplňte rok
Prague \RokOdevzdani

\end{center}

\newpage

%%% Here should be a bound sheet included -- a signed copy of the "master
%%% thesis assignment". This assignment is NOT a part of the electronic
%%% version of the thesis. DO NOT SCAN.

%%% A page with a solemn declaration to the master thesis

\openright
\hypersetup{pageanchor=true}
\pagestyle{plain}
\pagenumbering{roman}
\vglue 0pt plus 1fill

I~hereby declare that I~have authored this thesis independently,
and that all sources used are declared in accordance with the
\uvodzovky{Metodický pokyn o~etické přípravě vysokoškolských závěrečných prací}.

\vspace{2mm}
I~acknowledge that my thesis (work) is subject to the rights and obligations
arising from Act No. 121/2000 Coll., on Copyright and Rights Related to Copyright
and on Amendments to Certain Laws (the Copyright Act), as amended,
(hereinafter as the \uvodzovky{Copyright Act}), in particular §~35, and §~60 of the Copyright Act
governing the school work.

\vspace{2mm}
With respect to the computer programs that are part of my thesis (work)
and with respect to all documentation related to the computer programs (\uvodzovky{software}),
I~hereby grant the so-called MIT License.

\vspace{2mm}
The MIT License represents a~license to use the software free of charge.
I~grant this license to every person interested in using the software.
Each person is entitled to obtain a~copy of the software (including
the related documentation) without any limitation, and may, without limitation,
use, copy, modify, merge, publish, distribute, sublicense and/or sell
copies of the software, and allow any person to whom the software is further
provided to exercise the aforementioned rights. Ways of using the software or the extent
of this use are not limited in any way.

\vspace{2mm}
The person interested in using the software is obliged to attach the text of the license terms as follows:

\vspace{3mm}
\noindent
Copyright (c) \RokOdevzdani~\AutorPrace \\
Permission is hereby granted, free of charge, to any person \\
obtaining a copy of this software and associated documentation \\
files (the \uvodzovky{Software}), to deal in the Software without \\
restriction, including without limitation the rights to use, \\
copy, modify, merge, publish, distribute, sublicense, and/or sell \\
copies of the Software, and to permit persons to whom the \\
Software is furnished to do so, subject to the following conditions: \\
The above copyright notice and this permission notice shall be \\
included in all copies or substantial portions of the Software.

{
\footnotesize
\vspace{3mm}
\noindent
THE SOFTWARE IS PROVIDED \uvodzovky{AS IS}, WITHOUT WARRANTY OF ANY KIND, \\
EXPRESS OR IMPLIED, INCLUDING BUT NOT LIMITED TO THE WARRANTIES \\
OF MERCHANTABILITY, FITNESS FOR A PARTICULAR PURPOSE AND \\
NONINFRINGEMENT. IN NO EVENT SHALL THE AUTHORS OR COPYRIGHT \\
HOLDERS BE LIABLE FOR ANY CLAIM, DAMAGES OR OTHER LIABILITY, \\
WHETHER IN AN ACTION OF CONTRACT, TORT OR OTHERWISE, ARISING \\
FROM, OUT OF OR IN CONNECTION WITH THE SOFTWARE OR THE USE OR \\
OTHER DEALINGS IN THE SOFTWARE.
}

\vspace{15mm}

\hbox{\hbox to 0.5\hsize{%
In $\ldots\ldots\ldots$ date $\ldots\ldots\ldots$
\hss}}

\vspace{20mm}
\newpage

%%% Podekovani

\openright

\Podakovanie

\newpage

%%% Mandatory information page of the thesis

\openright

\vbox to 0.5\vsize{
\setlength\parindent{0mm}
\setlength\parskip{5mm}

Title:
\NazevPrace

Author:
\AutorPrace

\TypPracoviste:
\Katedra

Supervisor:
\Vedouci, \KatedraVedouciho

Abstract:
\Abstrakt

Keywords:
\KlicovaSlova

\vss}

\newpage

\openright
\pagestyle{plain}
\pagenumbering{arabic}
\setcounter{page}{1}



%%% Strana s automaticky generovaným obsahem bakalářské práce

\tableofcontents

%%% Jednotlivé kapitoly práce jsou pro přehlednost uloženy v samostatných souborech

\chapter{Introduction}



\section{Motivation}

Data lineage, or provenance, describes where data came from,
how it was derived and where the results are stored.

Lineage can be useful in a variety of settings. For example,
molecular biology databases, which mostly store copied data,
can use lineage to verify the copied data by tracking
the original sources [1]. Data warehouses can use the lineage
of anomalous view data to identify faulty base data [4],
and probabilistic databases can exploit lineage for
confidence computation [10, 12].
\TODO{Citacia - Data Lineage: A Survey}

In our work we focus on Java applications that access and/or transform data.
For these operations, often other frameworks are used to simplify
the application logic and communication with other systems.

In these frameworks, we need to get a graph, where nodes
contains sources and sinks of data and directed edges
are flows between them.

For example, when using database frameworks, in addition to
know how data flow from sources to sinks, we want to know
also what queries are executed to load and store the data.





\section{Goals}

We would like to create a library that meets the requirements of:

\begin{itemize}
  \item Correctness
  \item Accuracy
  \item Efficiency
  \item Support for few frameworks
  \item Easy to add support for new frameworks
\end{itemize}

Correctness and accuracy are the essential requirements of algorithms used
in our library. We have also efficiency in mind, so we precompute as much
as we could.

We implement \textit{plugins} for handling few frameworks, that can be used
in applications to show our approach works well. We chose different types
of frameworks, each with different access to data, so adding support
for new frameworks should be easy to implement - one should identify
the places where application works with data (load, store or just modify)
and algorithm symbolic analysis algorithm takes care of creating
data lineage part.



%Data can be accessed using databases, files, standard input and output,
%using network or many other ways.
%Many frameworks were created to manipulate with the data and
%we would like to develop library, which can identify places
%in these frameworks where data are accessed.


\section{Manta Flow}

Manta Flow is a tool that helps enterprises to get end-to-end
data lineage including custom SQL code. That allows customers to fulfill
compliance regulations or improve data governance.

It extracts and analyses metadata from report definitions,
custom SQL code, and ETL workflows, to create data flows
which span multiple systems and a range of technologies.
Lineages are analyzed based on actual code, and flows can be visualized.
Data paths between files, report fields, database tables,
and individual columns are revealed to users,
and this information can be utilized in a range of roles.

All entities detected by Manta Flow, such as columns or procedures,
are presented to users in a structure designed to simplify navigation.
Users can search for any object in the repository,
or within the visualization itself.
\TODO{Citacie: \\
https://www.capterra.com/p/145178/Manta-Flow/ \\
https://www.getapp.com/business-intelligence-analytics-software/a/manta-flow/ \\
https://reviews.financesonline.com/p/manta-flow/
}

As Manta flow does not work with Java code, data lineage
stops when reaching it. The support for Java can bind
previously separated data lineage parts done in databases
and user can get full data lineage when using Java and database
in application.

%- motivacia \\
%- kontext manty \\
%- DB frameworky v aplikaciach \\

%- co chceme \\
%-- funkcna analyza datovych tokov \\
%-- presnost, rychlost \\
%-- podpora analyzy pre db frameworky \\
%-- vytvorenie pluginov do analyzy pre niekolko frameworkov






\chapter{Úvod do statickej analýzy}

\section{Motivácia}

\section{MANTA}

\subsection{História}

História nástroja Manta Flow siaha do roku 2008, kedy ho vyvinula česká konzultačná spoločnosť Profinit s.r.o. ako interný nástroj.
Následne v roku 2013 sa autori projektu rozhodli založiť samostatnú spoločnosť a pokračovať v jeho vývoji.
V čele s Ing. Tomášom Krátkým založili spoločnosť Manta Tools s.r.o. známu ako MANTA.

Spoločnosť sa spolu s ČVUT zapojila do krantových programov ALFA a EPSILON Technologickej Agentury Českej Republiky (TAČR).
Vďaka nim získala v rokoch 2013 a 2017 dva granty v celkovej výške 1,49 milióna dolárov.

Spoločnosť sa presadila na zahraničnom trhu predovšetkým po výťazstve v súťaži ``Czech ICT Incubator @ Silicon Valley`` v roku 2014,
ktorú usporadúva Czech ICT Alliance. Po víťazstve bola založená prvá americká pobočka v San Francisku.

V súčastnosti pôsobí MANTA celosvetovo prostredníctvom vlastných pobočiek a siete regionálnych partnerov.
Medzi zákazníkov patrí napríklad spoločnosť Paypal, OBI, Vodafone, alebo Comcast.

O spoločnosti sa môžeme viac dočítať v článkoch od autorov \citet{MANTA} a \citet{MANTA_HISTORIA}.

\subsection{Manta Flow}

Manta Flow je nástrok umožňujúci automatickú analýzu programovacieho kódu (SQL, Java) a následný popis transformačnej logiky,
ktorý v ňom je obsiahnutý. Software je schopný rozpoznať aj ťažko čitateľné, na mieru napísané riadky programovacieho kódu.
Vďaka tejto vlastnosti dokáže v pomerne krátkom čase (obvykle niekoľko hodín) automaticky prečítať databáze o rozsahu
stotisícov i niekoľkých miliónov údajov a zostaviť z nich prehľadnú mapu datových tokov naprieč BI prostredím (Data Lineage).
To sa v praxi využíva najmä k optimalizácií datových skladov, znižovaniu nákladov na vývoj softwaru, vykonávanie dopadových
analýz a pri dokumentovaní prostredí pre potreby regulačných úradov.


% Prepinace pre figure:
% h=approximately here
% t=top of page
% b=bottom of page
% p=on special page
% H=precisely here
\newcommand{\InsertCode}[2]{\begin{figure}[#1]\input{#2}\end{figure}}

\newcommand{\Code}[1]{\texttt{#1}}

\chapter{Overview of Database Frameworks in Java \label{frameworks}}

\TODO{Nejaky iny nadpis, nejedna sa iba o DB frameworky - kafka}

In this chapter, we show basic features of chosen frameworks
that are used for manipulating data.
We focus on the basic features relevant for the data lineage
and ignore most of their advanced features.

We show basic examples of how data can be load or store by Java application
and how sources and sinks are identified, as this is the main topic of our work.

All frameworks that are using database, would work with example
of loading class \Code{DatabaseValue} from snippet \ref{code:model}
from database structured as in \ref{code:db}.

\InsertCode{h}{code/model}

\begin{table}[h]
  \centering
  \begin{tabular}{c c}
    \toprule
      \textbf{ID} & \textbf{VALUE} \\
    \midrule
      1 & A \\
      2 & B \\
      3 & C \\
    \bottomrule
  \end{tabular}
  \caption{Example of Database content}
  \label{code:db}
\end{table}







\section{JDBC \label{frameworks:jdbc}}

For accessing database in Java application, there exists standard Java Database Connectivity (JDBC) API,
which is described in more detail in \citet{JDBC_OVERVIEW}.

Database vendors ususally provide JDBC API implementation. API is generic, so
there should be no difference for connecting to different database types.

There are few interfaces in \citet{java.sql} package controlling database calls.
\begin{itemize}
  \item \Code{Connection}
  \item \Code{Statement}, \Code{PreparedStatement}, \Code{CallableStatement}
  \item \Code{ResultSet}
\end{itemize}

\Code{Connection} object should hold database connection and through this connection
database queries can be executed using any of \Code{Statement} calls.
When data are returned to application from statement, it is done through \Code{ResultSet}.

Getting connection to database is through \Code{DriverManager}, or from JDBC 2.0
it can be done using \Code{DataSource} and it is now the preferred way of connecting to database.
Example \ref{code:datasource} shows how \Code{DataSource} can be created for Oracle database
that is listening on url \Code{jdbc:oracle:thin:@//192.168.0.16:1521/orcl}
and \Code{User} user and \Code{Password} password is used when connecting to it.

Calls of \Code{createDataSource()} would be used in next examples.

\InsertCode{h}{code/datasource}
\InsertCode{H}{code/jdbc}

JDBC example \ref{code:jdbc} shows how can be done loading data from \Code{DataSource} (as in \ref{code:datasource}).
On line \ref{code:jdbc:connection}, connection to database is created.
Then on lines \ref{code:jdbc:prepareStatement:begin}--\ref{code:jdbc:prepareStatement:end}
database query is created to select just rows matching \Code{id} argument.
The query is then executed on line \ref{code:jdbc:executeQuery} and then
result is mapped from \Code{ResultSet} to our \Code{DatabaseValue} model and then it is returned on line \ref{code:jdbc:return}.

As you can see, there is huge amount of boilerplate code (catch-finally blocks) for closing every JDBC API object,
as exceptions can be thrown from almost all calls and we need to free all database resources that we do not need anymore.

From Java 7, try-with-resources can be used with result, that all finally blocks can be removed - resources
(Connection, PreparedStatement, ResultSet) are automatically closed after finishing block.
This is ilustrated in example \ref{code:jdbc-try-with-resources}.

\InsertCode{h}{code/jdbc-try-with-resources}




\section{Spring JDBC Framework \label{frameworks:jdbcTemplate}}
\citet{SpringJDBC} Framework is extension above JDBC API and tries to help users to code only
parts with application logic and it removes much of the boilerplate code.
From next list, only italicized lines need to be coded by user:
\begin{itemize}
  \item Define connection parameters
  \item Open the connection
  \item \textit{Specify the statement}
  \item Prepare and execute the statement
  \item Set up the loop to iterate through the results (if any)
  \item \textit{Do the work for each iteration}
  \item Process any exception
  \item Handle transactions
  \item Close the connection   
\end{itemize}

Before Java 7, coding in standard JDBC tend to be errorneus because of forgetting to close
database resources and boilerplate code does not help in readability of code
(as could be seen in example \ref{code:jdbc}). These problems are removed using Spring JDBC Framework,
as illustrate example \ref{code:jdbcTemplate}. It shows, how can be loaded single row from database.
We can see, that mappings are the same in standard JDBC and Spring JDBC examples.
Framework handles all boilerplate code around and result is returned after processing single line \ref{code:jdbcTemplate:return}.


\InsertCode{h}{code/jdbcTemplate}





\section{MyBatis \label{frameworks:myBatis}}

\citet{MyBatis} framework is one of Object-Relational Mapping (ORM) frameworks.
It uses JDBC API to communicate with database, but in almost all cases, there is no need
to work with low level JDBC.

Unlike other ORM frameworks, it does not map Java objects to database tables, but Java methods
to SQL statements. All communication with database is always through methods in user created interfaces.
SQL statements are stored in XML files or annotations in these interfaces.

In subsection \ref{mybatis:mapper} we show examples of \Code{Mapper} interface definitions,
which can be done by annotations or using XML mapper files.
Logic that is made in background by MyBatis by calling these interfaces
is same as we could see in previous JDBC or Spring JDBC examples,
when from some database connection \Code{PreparedStatement} is created.
Next, method argument \Code{id} is set and query is executed.
After execution, from \Code{ResultSet} are values of columns
mapped to object attributes and object is returned.

Subsection \ref{mybatis:configuration} shows how database
configuration can be made and subsection \ref{mybatis:run}
shows how data can be loaded from such a database.



\subsection{MyBatis Mapper definition \label{mybatis:mapper}}

\InsertCode{h}{code/mybatis-interface-annotations}

Example \ref{code:mybatis:interface:annotations} uses annotations to store definitions of both query and mapping.
Query is defined using \Code{@Select} annotation on line \ref{code:mybatis:interface:annotations:query}
and mapping is defined using \Code{@Results} and \Code{@Result} annotations.

\InsertCode{h}{code/mybatis-interface-xml}
\InsertCode{H}{code/mybatis-mapper-xml}

Snippets \ref{code:mybatis:interface:xml} and \ref{code:mybatis:mapper:xml} contains definitions
of plain interface which has query and mapping stored in XML mapper file.
Query is located on line \ref{code:mybatis:mapper:xml:query} in \Code{<select>} tag.
Tag contains reference to correct \Code{resultMap} mapping on lines
\ref{code:mybatis:mapper:xml:mapping:begin}--\ref{code:mybatis:mapper:xml:mapping:end}.



\subsection{MyBatis Configuration \label{mybatis:configuration}}

MyBatis framework uses interface \Code{SqlSessionFactory} for creating database connections.
Factory can be created using both Java and XML files.

\InsertCode{h}{code/mybatis-sessionFactory-java}

Example \ref{code:mybatis:sessionFactory:java} shows how configuration can be done in Java.
We set \Code{DataSource} reference to \Code{Environment} class on line \ref{code:mybatis:sessionFactory:java:dataSource}.
\Code{JdbcTransactionFactory} was used to handle database transactions.
On line \ref{code:mybatis:sessionFactory:java:addMapper} mapper class is registered to be known by MyBatis
and next \Code{SqlSessionFactory} is created.

\InsertCode{h}{code/mybatis-sessionFactory-xml}
\InsertCode{H}{code/mybatis-configuration-xml}

The same can be also done using XML configuration file.
\Code{SqlSessionFactory} is created in snippet \ref{code:mybatis:sessionFactory:xml}
after loading configuration file on line \ref{code:mybatis:sessionFactory:xml:file}.

XML configuration file \ref{code:mybatis:configuration:xml} configures \Code{DataSource} on lines
\ref{code:mybatis:configuration:xml:dataSource:begin}--\ref{code:mybatis:configuration:xml:dataSource:end}
and \Code{Mapper} class is registered on line \ref{code:mybatis:configuration:xml:mapper}.



\subsection{Loading data from database using MyBatis \label{mybatis:run}}

As we know how to configure database connections and how to define mappers,
in example \ref{code:mybatis} we show how data can be loaded.

On line \ref{code:mybatis:getMapper}, the implementation of \Code{Mapper} interface
that was made by MyBatis is returned and on next line \ref{code:mybatis:return} database query is executed.

\InsertCode{h}{code/mybatis}



\section{Kafka \label{frameworks:kafka}}

Apache Kafka \citet{Kafka} is a distributed streaming platform.
It means, that application can publish or subscribe records and
process them, as they occurs.

Application that want to publish some records, it sends them to server
and data are send to all subscribers of the same \textit{topic}
(for detailed info see section \ref{frameworks:kafka:background}).

Kafka has four core APIs:
\begin{itemize}
  \item The \textbf{Producer API} allows an application to publish
    a stream of records to Kafka topics
  \item The \textbf{Consumer API} allows an application to subscribe
    to topics and process the stream of records produced to them
  \item The \textbf{Streams API} allows an application to act as a stream processor,
    consuming an input stream from topics and producing an output stream to output topics,
    effectively transforming the input streams to output streams
  \item The \textbf{Connector API} allows building and running reusable producers or consumers
    that connect Kafka topics to existing applications or data systems.
    For example, a connector to a relational database might capture every change to a table. 
\end{itemize}

Figure \ref{frameworks:kafka:api} shows, how client applications can communicate with Kafka server
using Kafka APIs.

\begin{figure}[h]
  \center
  \includegraphics[width=100mm]{img/kafka-apis.png}
  \label{frameworks:kafka:api}
  \caption{Applications using differend kinds of Kafka APIs}
\end{figure}



\subsection{Kafka background \label{frameworks:kafka:background}}

Kafka can be run as a cluster on one or more servers.
Each cluster stores stream of records in categories that are called topics
and each record consist of a key, value and timestamp.

Topics can be compared to the database tables. Application that is publishing topic
writes to that table and the one that subscribe topic reads that data.

Kafka topics can be also partitioned, so distributed computations can be made
on them - each partition can be handled by different server/producer/consumer.
Each partition is an ordered, immutable sequence of records that is
continually appended to.

The Kafka cluster durably persists all published records (whether or not they have been consumed)
using a configurable retention period. For that period, any consumer can access
to any record published, as can be seen in figure \ref{frameworks:kafka:partitions}.

\begin{figure}[h]
  \center
  \includegraphics[width=100mm]{img/kafka-partitions.png}
  \label{frameworks:kafka:partitions}
  \caption{Producers are appending new records to Kafka partition and Consumers accessing them}
\end{figure}



\subsection{Kafka Producers and Consumers}

\InsertCode{h}{code/kafka-producer}
\InsertCode{h}{code/kafka-consumer}

Structure of data in records can be arbitrary. Application just need to handle
correct transformation of used Java object to (or from) byte array using
\Code{Serializer<T>} (or \Code{Deserializer<T>}) objects.
However, for our data lineage problem, feature of serializing
and deserializing objects is unimportant. There is always one
source/target - the byte array. There, we cannot distinguish values
that belong to same attribute that was written to and then read from that array\footnote{
  As could be done in case of databases, where rows were divided into columns.
}.

Using Producer API, application can create new records and send them to topic of Kafka server.
After creating \Code{KafkaProducer} with some configuration as in \ref{code:kafka:producer},
records can be send to server by calling \Code{send()} method.

With Consumer API, application can handle new records that are arriving from server.
One has to create configured \Code{KafkaConsumer}, subscribe to some topics
and wait for new records. This is illustrated in snippet \ref{code:kafka:consumer}.
On line \ref{code:kafka:consumer:properties}, kafka server url is configured.
Then consumer registers itself to receive records in topic \Code{Topic}
on line \ref{code:kafka:consumer:subscribe} and on line \ref{code:kafka:consumer:poll}
kafka is queried for data. There is some maximal time limit, for which kafka waits for
new records to arrive (in our case 1 second) and then it returns them.



\subsection{Kafka Streams}

A Kafka Stream represents an unbounded, continuously updating data set.
A stream is an ordered, replayable, and fault-tolerant sequence of immutable data records,
where a data record is defined as a key-value pair.
Application defines its computational logic through processor topologies,
where processor topology is a graph of stream processors (nodes) that are
connected by streams (edges).
Stream processor is a node in the processor topology, that represents
a processing step to transform data in streams by receiving one
input record at a time from its upstream processors in topology,
applying its operation on it and then produce one or more
output records to its downstream processors.

In topology, there are two special processors:
\begin{itemize}
  \item \textbf{Source Processor}: A source processor is a stream processor
    that does not have any upstream processors. It produces an input stream
    to its topology from one or multiple Kafka topics by consuming records
    from these topics and forwarding them to its down-stream processors, and
  \item \textbf{Sink Processor}: A sink processor is a stream processor
    that does not have down-stream processors. It sends any received records
    from its up-stream processors to a specified Kafka topic.
\end{itemize}

The way, how topology can be defined is using the Kafka Streams DSL (Domain Specific Language).
It provides the most common data transformation operations, such as
\Code{map}, \Code{filter}, \Code{join} and \Code{aggregations}.
There exists also the low level Processor API, that allows developers define
and connect custom processors and also interact with state stores.




\subsection{Kafka Connector}

Kafka Connect is a tool for scalably and reliably streaming data between Apache Kafka
and other systems. It makes it simple to quickly define connectors that move
large collections of data into and out of Kafka.
Kafka Connect can ingest entire databases or collect metrics from all
your application servers into Kafka topics, making the data available
for stream processing with low latency.




\chapter{Static Analysis \label{chapter:analysis}}

\TODO{ \\
  -- Co je staticka analyza \\
  -- Rozdiel od dynamickej verzie \\
  -- Naco a kedy sa vyuziva \\
  -- Ake su jej obmedzenia \\
  -- Ako ju mozeme vyuzit v nasej praci \\
}

In this chapter, we will introduce concept of static analysis,
its possible applications and its limitations.

\section{Program Analysis}

\textbf{Program analysis} is the process of automatically analyzing the behavior
of computer programs. There are two approaches of such analysis:
\begin{itemize}
  \item \textbf{Dynamic program analysis} is performed during program runtime.
    To perform dynamic program analysis, both executable program
    and its inputs are required. To be effective, the target program
    must be executed with sufficient test inputs, as its results
    are limited only to observed executions of the analyzed program.
  \item \textbf{Static program analysis} is the program analysis that is actually
    performed without executing program. The analysis is usually
    performed on the program source code, or bytecode.
    Results of static program analysis include all execution branches of
    analysed program.

    Static program analysis technique is very popular. It is thanks
    to its speed, reliability and sometimes it is the only possible
    way for complex systems in reasonable time.
    It is is often used to detect program errors, such as
    security vulnerabilities or performance optimizations.
\end{itemize}

\textbf{Data flow analysis} is a technique for gathering information about
all possible values of variables during program execution.



\section{WALA Framework \label{chapter:analysis:wala}}

The T. J. Watson Libraries for Analysis, the \citet{WalaFramework},
is a framework for static and dynamic analysis capabilities for Java
bytecode and related languages.

The main goals of WALA Framework are:
\begin{itemize}
  \item Robustness
  \item Efficiency
  \item Extensibility
\end{itemize}

The key features WALA Framework provides are:
\begin{itemize}
  \item Pointer analysis
  \item Class hierarchy
  \item Call graph
  \item Interprocedural dataflow analysis
  \item Context-sensitive slicing
\end{itemize}

We describe some of the features in next sections.



\subsection{Pointer analysis}

\citet{PointerAnalysis} define \textbf{pointer analysis},
or \textbf{points-to analysis} respectively, as a static program analysis that
determines information on the values of pointer variables or expressions.
It is near-synonym of \textbf{alias analysis} that use \citet{AliasAnalysis}.
Pointer analysis typically answer question
\uvodzovky{what objects can a variable point to?},
whereas alias analysis focus on closely related question
\uvodzovky{can a pair of variables point to the same object?}.




\subsubsection{Flow and Context Sensitivity}

Flow sensitivity refers to ability of an analysis to take control flow
into account when analyzing a program.
In case, analysis considers statement ordering, it is called \textbf{flow-sensitive},
otherwise it is \textbf{flow-insensitive} analysis.

Contex sensitivity can be taken into account in an interprocedural analysis and
then we call it \textbf{contex-sensitive}. Otherwise, when calling context is ommited,
such an analysis is called \textbf{context-insensitive}.




\subsubsection{Andersen's algorithms}

\TODO{Popisat, ak ho pouzivame aj v nasej analyze: \\
http://wala.sourceforge.net/wiki/index.php/UserGuide:PointerAnalysis}




\subsection{Class hierarchy}

A \textbf{class hierarchy} is a structure for set of classes
of the analyzed program. There is also information about the used
programming language and the relationships between such classes.

Such relations in programs written in Java language are \Code{implements} and \Code{extends}
relations.

For each class, there is also collection of methods. Methods can be declared in the
class, or inherited from parent.



\subsection{Call Graph}

A \textbf{call graph} represents calling relationships between subroutines
in an analysed computer program. The nodes represents procedures and each
directed edge represents procedure calls.

When used in context-sensitive manner, which means that for each procedure 
the graph contains a separate node for each call stack that procedure can be
activated with.




\section{Symbolic Analysis Library}

\textbf{Symbolic analysis library} is a library for computing flow of information
in Java program - the data lineage.
Symbolic analysis library was introduced in \citet{ParizekHybridAnalysis}
and in the research in that field continues and its results will be published
in a short time in \citet{ParizekBUBEN}.

The library uses static analysis techniques to construct call graph and
for each invocation context (parameter values) of method,
it computes its summary based on symbolic bytecode interpretation.

The bytecode analysis is done using \citet{WalaFramework}
that was described in section \ref{chapter:analysis:wala}.

Bytecode interpreter performs linear traversal of the method bytecode instructions
and computes various information for symbolic variables and constant expressions.

The \citet{ParizekBUBEN} defines method summaries as a data structure
that contains the following information:
\begin{itemize}
  \item A list of all object fields possibly updated in the method.
  \item A list of all possibly updated static fields.
  \item A list of all array elements updated in the method.
  \item For each updated field and array element, a list of possible new values.
  \item A boolean flag saying whether the method may return a value corresponding
    to its arguments, and indexes of the respective arguments.
  \item A set of all objects newly allocated in the method, including arrays.
  \item A boolean flag saying whether the method may return a new object.
  \item A set of all possible return values.
  \item A list of fields and array elements updated outside of any code region that is
    protected by a lock or through another mechanism of thread synchronization.
\end{itemize}



\subsection{Symbolic Analysis Algorithm}

The symbolic analysis algorithm for computing static method summaries use:
\begin{enumerate}
  \item A \textbf{fixpoint worklist algorithm} over the list of all reachable methods.
  \item A \textbf{linear symbolic interpretation} of the bytecode of methods.
\end{enumerate}

The library is iteratively updating its method summaries until the fixpoint is reached, i.e.
when summaries are not changing.

Several method invocation contexts (method parameters with their associated data flow information)
are distinguished. When the newly computed summary for a given method is different from the previous
one, all callers and calles of the method are added to the worklist in order to achieve soundness.
The algorithm terminates when the summary of each method captures all its results and side effects.




\subsection{Flow Propagation}

In the previous section we described general algorithm for symbolic analysis.
However, we did not specify, how the flow information is used in practice.




\subsubsection{Library Methods}

As optimizations, library does not process all reachable methods.
It analyses just methods of application code and procedures with special
meaning, that are described in next sections.

For ignored methods, the \textbf{identity} is returned.
The identity function with respect to flow data propagation is defined as
merge of flow data of receiver (the object on which method is called) and
all arguments of that method. The result of that merge is then associated
as flow data of returned value and receiver.




\subsubsection{Strings}

Know the concrete values of the \Code{String} used in application is valuable
when it comes to data lineage of application.
\Code{String}s identify file names that are accessed in application,
or it can be query to database, etc.

However, concrete values are known only when using as literal in the application.
Sometimes, the symbolic analysis cannot determine the precise concrete string value.
It is often when it is loaded from file or database, or after some operations
like \Code{trim}, \Code{substring}, or even concatenation of \Code{String}s.
Then the actual value is in general unknown, but the flow information must be preserved.




\subsubsection{Numeric types}

The idea of concrete values for numeric Java types works as in case
of \Code{String} literals. It is useful to know them, but often
they cannot be determined. Flow information must be preserved after
the operations with the numbers in any case.




\subsubsection{Arrays and Collections}

Arrays and collections are integral parts of Java. In this part we describe how
they are handled in propagation of flow.

The basic approach is not to distinguish individual items and use just one
abstract element summary. In that case, as over-approximation, all possible elements
of a given array or collection are considered.
Complete flow information is also used when creating \Code{java.util.Iterator}
from a collection.

Also, when using \Code{java.util.Map} and \Code{java.util.Properties} classes,
analysis does not distinguish between used keys and corresponding values
and maintains just a single set of flow for the whole class.

When collections of \Code{String}s or numeric values library provides
the information about concrete values stored in collections whenever possible.





\subsection{Identifying the Sources and Sinks of a Data}

Symbolic analysis algorithm, that we describe in previous section is
used to compute data flow in application.
When such sources or sinks are identified, the algorithm would
propagate such data flow.

Now, the problem is to identify the sources and sinks of a data
in the application.




\subsubsection{Java IO}

Java input and output (IO) operations can be used to access data in application.
It can be done through standard application
\Code{System.in}, \Code{System.out} and \Code{System.err},
where they are identified by the name \uvodzovky{System.in}, etc.
or using external files, where its file name identifies the source (or sink respectively).

The library handle both cases and correctly identify
all the read and write operations of such inputs and outputs.

The library focus only to inputs and outpus made by classes in \Code{java.io} package
but also the \Code{java.nio} package would be supported soon.




\subsubsection{JDBC API}

The library also contains implementation for identification of
database reads and writes using plain JDBC API.
The related usage of the JDBC API was already described in section \ref{frameworks:jdbc}.

The library can identify the connection url for the database
that application is connecting to, SQL queries and the columns,
that are read from results in \Code{ResultSet}.

The library focus only to classes in package \Code{java.sql}
therefore \Code{javax.sql} API is not supported.



\chapter{Implementation of Data Lineage for Frameworks}

In this chapter, we will present our solution of Data Lineage library for Database Frameworks.
We will discuss different problems that came with all the frameworks presented in chapter \ref{frameworks}.

\section{Interface}

\section{JDBC \label{implementation:jdbc}}

Implementation of Data Lineage for standard JDBC is done using the library.

\TODO{Nieco o JDBC}

\TODO{DataSource implementacia}



\section{Spring JDBC Framework}

Spring JDBC Framework comes with approach of \Code{Callbacks}. Library handles
all boilerplate code and calls user defined callback objects to do its job
and after finishing it, library handles closing all resources that is not needed anymore.

When we looked to implementation of \Code{JdbcTemplate} class, we found out
that database calls are made only by 4 out of about 50 its methods.
All other methods call them to execute some queries.

\TODO{Pokracovanie JdbcTemplate}



\section{MyBatis}

In chapter \ref{frameworks:myBatis} we showed classic use case of loading data from database.
Now, we present solution of creating data lineage for this framework.



\subsection{Mapper interfaces}

To find all the mappers, we iterate all classes in WALA \Code{IClassHierarchy}.
We are searching for interfaces with methods annotated with MyBatis annotations,
or interfaces for which XML mapper definition exists.



\subsubsection{XML mapper files}

XML mapper file has same name as interface (but .xml extension is used instead of .java),
and layes in same directory.

We made a parser of such a XML files, that can get information about SQL statements
and column to property mapping.
MyBatis provide rich feature set for XML mappers.
We will describe how mappers are defined and some of the advanced features
that we are handling.

Basic mapper, which example was showed in \ref{code:mybatis:mapper:xml} contains of SQL statement
with defined \Code{ResultMap} when needed. MyBatis use tags
\Code{<select>}, \Code{<insert>}, \Code{<delete>} and \Code{<update>}
for storing SQLs and \Code{<resultMap>} for mapping which is divided into column-to-property
mapping pairs.

From more advanced features, MyBatis supports reusable fragmets. It means that
one can define some fragment of SQL and reuse it in more queries.
Each fragment has its own ID, so it is easy to find correct one and include it into query.

Another feature is dynamic SQLs, where queries are created dynamically
as some conditions hold. As we do not know which conditions are valid
at runtime, we made simplification that we always use first branch
that we see in code.
\TODO{Ukazat priklad kodu a jeho vystup}



\subsubsection{Annotated mapper classes}

When mappers use annotations, it is quite simple to get all data needed
from WALA. We know, that SQL statement is always stored in annotations
\Code{@Select}, \Code{@Insert}, \Code{@Delete} or \Code{@Update}
and queries are stored in plain \Code{String} array.
Queries can also contain dynamic SQL as in XML definitions.

For mapping result object the \Code{@Results} is used with \Code{@Result}
for every column to object property mapping definition.
\TODO{Pripadne pridat aj @ConstructorArgs, kedby to bolo hotove}

\TODO{Co vystupne argumenty metody?}

There also exists more advanced features of MyBatis when using annotations\footnote{
  Such as using provider classes to create SQL queries (\Code{@SelectProvider}, $\ldots$),
  generating IDs from sequence (\Code{@SelectKey}) and using them in queries,
  generating maps from objects (\Code{@MapKey}),
  associations for attribute classes (\Code{@One} or \Code{@Many}),
  mixing XML and annotation mappers (e.g. define \Code{<resultMap>} in XML and reference it
  using \Code{@ResultMap} annotation)}
but we do not handle them. All of them can be added as new functionality in next development,
if needed.




\subsection{Database connection configuration}

Configuration of MyBatis can be done in two ways, using XML configuration file
or in Java using \Code{Configuration} class. Second way needs no more attention,
as we use \Code{DataSource} classes that are already handled in section \ref{implementation:jdbc}.

To handle XML configuration, we need parse configuration file and find \Code{<dataSource>}
section and its properties \Code{driver}, \Code{url} and \Code{username}. All this
information we need for identifying target database.



\section{Kafka}




\chapter{User Documentation for the \ToolName Tool \label{chapter:program}}

We created a set of plugins for Symbolic analysis library
for extracting data lineage information from data processing frameworks used in Java applications.
We also provide some test scenarios for frameworks we are handling.

We created the \ToolName tool, which is composition of the
Symbolic analysis library together with the plugins.

Here we describe installation and usage of the \ToolName tool.
We use standard Maven directory structure for source code.
In \Code{src/main} directory, application source and resource files
are located and in \Code{src/test} test sources and resources
are located.



\section{Software Requirements}

To successfully install the \ToolName tool, following is required:

\begin{itemize}
  \item Java JDK 1.8 installed
  \item Apache Maven installed and configured
    \begin{itemize}
      \item Recommended version is 3.3.9
    \end{itemize}
  \item Graphviz installed with tool \Code{dot} on PATH
    \begin{itemize}
      \item Recommended version is 2.40.1
    \end{itemize}
\end{itemize}

We recommend to use such versions of the software,
as these versions were used for testing and some compatibility
issues can be encountered using different versions.



\section{Installation}

To compile the \ToolName tool and create runnable \Code{JAR} file, use the command:
\begin{lstlisting}[language=shell]
        # mvn clean compile assembly:single
\end{lstlisting}

We recommend also to run all tests after instalation by:
\begin{lstlisting}[language=shell]
        # mvn test
\end{lstlisting}
When tests ends, the flow graph visualizations are saved in the \Code{target/img} directory.

As it could take a long time to run all tests (more than an hour), we provide TestNG
suites for each framework. This can be useful to verify that requested plugin is working
before trying to use it.

The test suite files are located in subdirectories of \Code{test/resources/tests}
and can be run using command:
\begin{lstlisting}[language=shell]
        # mvn test -DtestSuite=<suiteFile>
\end{lstlisting}


\section{Running the \ToolName Tool}

After instalation is completed, the \ToolName tool is ready to be used for computing
data lineage of target application using command line options:

\TODO{nazov jarka; skontrolovat este raz argumenty a nazvy prepinacov, ci sa nezmenili}

\begin{lstlisting}[language=shell]
        # java -jar library.jar [OPTIONS]

        OPTIONS:
          [--application-jar <fileName>]
          [--library-jar <fileName>]
          [--application-package <package>]
          [--entry <className> <methodSignature>]
          [--output-directory <directoryName>]
          [--help]
\end{lstlisting}

\begin{itemize}
  \item \Code{-{}-aplication-jar} specifies the application JAR file
    to be analysed by our \ToolName tool.
  \item \Code{-{}-library-jar} specifies the library JAR file
    to be known by our \ToolName tool.
    All library dependencies should be added for analysis.
  \item \Code{-{}-aplication-package} sets the root package of analysed application.
    This is needed, because of optimizations, when our \ToolName tool
    does not analyse classes that are outside of that package.
  \item \Code{-{}-entry} specify an entry point for the analysis.
    It can be some method, from which data lineage is computed.
    Fully qualified name of class should be provided and only methods
    without arguments and standard Java main method are supported.
    \TODO{tiez WALA method signatura}
  \item \Code{-{}-output-directory} specifies where the result files are stored.
  \item \Code{-{}-help} display usage and exit.
\end{itemize}




\section{Result Graphs \label{chapter:program:graphs}}

The data lineage of an input program is represented by a flow graph,
where nodes are data sources and sinks and oriented edges are between
pair of nodes between which the data flows.

The graph can be visualized using Symbolic analysis library visualization tool
described in Section \ref{chapter:analysis:visualization}.

The \ToolName tool generates three types of files. First, the \Code{.dot} file contains graph definition
from which the \Code{dot} tool creates \Code{.pdf} and \Code{.svg} files with visualized result graph.

The result visualization was described in Section \ref{chapter:analysis:visualization}
for JDBC and I/O APIs.
Now we continue the list of node types from that section
that are relevant to Symbolic analysis library plugins:
\begin{itemize}
  \item \Code{FrameworkDataSource} - defines source of data in framework (like SQL query statement).
  \item \Code{FrameworkDataField} - defines resulting field from framework query.
  \item \Code{FrameworkAction} - defines the data sink (like SQL insert statement).
\end{itemize}

Plugins define also own attributes for identification of some valuable information
for data lineage.
Such attribute can be \Code{KAFKA\_TOPIC} that is used for identification of used
topic in Kafka Framework, or \Code{FILE\_NAME} that identifies a file name that was
used as some input in program, like the configuration file in MyBatis.
There are also much more attributes.




\section{Implementing Support for new Framework}

We implemented support for three data processing frameworks,
but in general, any framework can be handled.
It can be done by implementing a new plugin for Symbolic analysis library.

In Section \ref{chapter:implementation:interface} we described interface between
Symbolic analysis library and its plugins. Plugins should identify the data sources and sinks
and the library take care of the data flow between them.

We propose first to make familiar with already implemented plugins and other
classes used in that plugins:
\begin{itemize}
  \item \Code{JdbcTemplateAnalysisPlugin} and \Code{JdbcTemplateHandler} can be inspiration for the approach of callbacks.
  \item \Code{MyBatisAnalysisPlugin} can be inspiration, when plugin should work with external files or annotations.
    \begin{itemize}
      \item \Code{MyBatisAnnotationMapperSqlReader} - working with annotations
      \item \Code{MyBatisXmlMapperSqlReader} - working with external XML files
    \end{itemize}
\end{itemize}



\chapter{Evaluation \label{chapter:results}}

In this chapter, we discuss the results of our \ToolName tool.
The tool provides implementation of plugins for three frameworks
that are used for manipulating data - Spring JDBC, MyBatis and Kafka.
It also provides implementation for the identification of many types of databases
through the \Code{DataSource} interface.

For the \ToolName tool we created a comprehensive set of tests.
They tests major part of the features of the tool and show
that our tool is in a good condition.

The \ToolName tool can also visualize data flow of JDBC and I/O API that was
implemented in the Symbolic analysis library and is not part of our work.

In the next sections, we evaluate results of our \ToolName tool for each of the selected frameworks.
At the end, we discuss limitations of the current solution and fulfillment of
the requirements from Section \ref{frameworks:requirements}.




\section{Data Flow Graph Visualizations for Frameworks}

In this section we present some data flow visualizations
for example programs that are presented in tests for the \ToolName tool.

\TODO{Obrazky z testov}



\section{Limitations of the \ToolName Tool \label{chapter:results:limits}}

There are few limitations of our \ToolName tool.
We point them out in the following list:
\begin{enumerate}
  \item \textbf{Slow data lineage computation.} \\
    The data lineage computation tests in our project last from
    a few tens of seconds to a few minutes. It depends on the
    number of used libraries, their complexity and also from the complexity
    of analysed application.
  \item \textbf{There can be many algorithm iterations until fixpoint is reached.} \\
    The Symbolic analysis library computes method summaries iteratively until
    fixpoint is reached - when there are no changes in the method summaries.
    The number of iterations also depends on the number of methods in the analysed program.
  \item \textbf{Some inaccuracies can occur based on the used\break over-approximation algorithm.} \\
    The Symbolic analysis library computes over-approximate data lineage information (e.g. when accessing
    an element of an array or collection, all elements are considered as output).
  \item \textbf{Disadvantages of static analysis.} \\
    Many disadvantages come with the usage of static analysis
    (e.g. when values are computed dynamically based on the external inputs).
\end{enumerate}



\section{Fulfillment of the Requirements}

In the Section \ref{frameworks:requirements} we pointed out the requirements
on the \ToolName tool to be able to successfully compute the data lineage of
applications that use frameworks for accessing the data.

In the next list we try to explain how the requirements are fulfilled (or not)
by the \ToolName tool:
\begin{enumerate}
  \item \textbf{Analysis of whole Java application.} \\
    Our tool analyses both application and its dependencies
    to distinguish all the data flows. The used WALA framework
    needs all dependencies to create a correct call graph.
  \item \textbf{Identifying the data sources and sinks.} \\
    The plugins for Symbolic analysis library identify the data sources and sinks
    of an analysed application when the corresponding frameworks are used.
  \item \textbf{Correctness, accuracy and efficiency of computing data lineage.} \\
    We explained some limitations of our \ToolName tool in Section \ref{chapter:results:limits}.
    The data lineage computation can take quite long time, depending on the complexity of an analysed
    application and the number of used libraries.
    However, it is expected that analysis of a huge program with many library dependencies
    will run longer than the analysis of other small Java application.
    As the tests in the \ToolName tool show, the data lineage is computed
    in few minutes, therefore the tool is usable in practice.
  \item \textbf{Work with external files.} \\
    The external configuration files are often handled by our plugins for the Symbolic analysis library.
  \item \textbf{Use of concrete values.} \\
    The Symbolic analysis library always tries to resolve actual values of Java primitives and
    \Code{String}s. However, even if in many cases the concrete values cannot be determined,
    the data flow between sources and sinks are preserved.
    As the frameworks often use external files for storing its data, that files can be often read
    by the \ToolName tool to identify the data sources and sinks.
  \item \textbf{Handle callbacks.} \\
    The Symbolic analysis library supports also computing the data flow in callbacks
    from the frameworks to applications.
  \item \textbf{Easy extendability.} \\
    We proposed the pluggable architecture for the Symbolic analysis library.
    We demonstrated the possibilities that result from the changed architecture
    on several library plugins. The plugins identify the data sources and sinks
    of used data processing frameworks.
\end{enumerate}


\chapter*{Záver}
\addcontentsline{toc}{chapter}{Záver}

TODO Nejaky zaver


%%% Seznam použité literatury
\include{literatura}

%%% Obrázky v bakalářské práci
%%% (pokud jich je malé množství, obvykle není třeba seznam uvádět)
\listoffigures

\lstlistoflistings
\addcontentsline{toc}{chapter}{List of Listings}


%%% Tabulky v bakalářské práci (opět nemusí být nutné uvádět)
%%% U matematických prací může být lepší přemístit seznam tabulek na začátek práce.
%\listoftables

%%% Přílohy k bakalářské práci, existují-li. Každá příloha musí být alespoň jednou
%%% odkazována z vlastního textu práce. Přílohy se číslují.
%%%
%%% Do tištěné verze se spíše hodí přílohy, které lze číst a prohlížet (dodatečné
%%% tabulky a grafy, různé textové doplňky, ukázky výstupů z počítačových programů,
%%% apod.). Do elektronické verze se hodí přílohy, které budou spíše používány
%%% v elektronické podobě než čteny (zdrojové kódy programů, datové soubory,
%%% interaktivní grafy apod.). Elektronické přílohy se nahrávají do SISu a lze
%%% je také do práce vložit na CD/DVD.
%\chapwithtoc{Prílohy}

\openright
\end{document}
