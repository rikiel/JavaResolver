
\chapter{Prehľad používaných frameworkov na komunikáciu s databázou}

V tejto kapitole popíšeme techniky a niektoré frameworky, ktoré umožňujú aplikácii prístup k rôznym druhom databáz.

Prácu s frameworkami si budeme ilustrovať na získaní dát pre triedu Category \ref{code:category} z databáze.


\begin{lstlisting}[caption={Príklad objektu získavaného z databáze}, label={code:category}]
public class Category {
    private Integer id;
    private String name;
    private String description;

    public Integer getId() {
        return id;
    }

    public void setId(Integer id) {
        this.id = id;
    }

    public String getName() {
        return name;
    }

    public void setName(String name) {
        this.name = name;
    }

    public String getDescription() {
        return description;
    }

    public void setDescription(String description) {
        this.description = description;
    }
}
\end{lstlisting}


\section{JDBC}

Prístup aplikácie k databáze je v Jave štandardne cez Java Database Connectivity (JDBC) API.
Rozhranie je implementované pre rôzne typy databáz. Dovoľuje pristupovať k dátam priamo s použitím Javy.
Architektúra pre prístup k databáze je popísaná v článku \citet{JDBC_OVERVIEW}.

Komunikácia s databázou sa uskutočňuje pomocou rozhraní a tried z balíka \citet{java.sql}.
Niektoré zo základných štruktúr si dôkladnejšie popíšeme.

\begin{itemize}
  \item Connection - objekt zabezpečujúci pripojenie do databáze
  \item Statement, PreparedStatement, CallableStatement - objekty obaľujúce jednotlivé volania databáze
  \item ResultSet - objekt obaľujúci výsledok volaní databáze
\end{itemize}

Na príklade \ref{code:jdbc} si ukážeme načítanie triedy Category z Oracle databáze.
Riadok \ref{code:jdbc:forName} zaregistruje Oracle ovládač databáze.
Následne riadok \ref{code:jdbc:getConnection} vytvorí pripojenie k databáze,
riadok \ref{code:jdbc:executeQuery} vykoná sql príkaz a na ďalších riadkach sa z výsledku získajú
hodnoty zo stĺpcov ID, DESCRIPTION a NAME.


\begin{lstlisting}[caption={Example of loading data from Oracle database using JDBC API}, label={code:jdbc}]
public DatabaseValue getForId(int id) throws SQLException {
 final OracleDataSource dataSource = new OracleDataSource(); /*\label{code:jdbc:dataSource:begin}*/
  dataSource.setURL("jdbc:oracle:thin:@//192.168.0.16:1521/orcl");
  dataSource.setUser("User");
  dataSource.setPassword("Password"); /*\label{code:jdbc:dataSource:end}*/

  Connection connection = null;
  try {
    connection = dataSource.getConnection();
    PreparedStatement preparedStatement = null;
    try {
      String query = "SELECT ID, VALUE FROM T WHERE ID = ?";
      preparedStatement = connection.prepareStatement(query); /*\label{code:jdbc:prepareStatement:begin}*/
      preparedStatement.setInt(1, id); /*\label{code:jdbc:prepareStatement:end}*/
      ResultSet resultSet = null;
      try {
        resultSet = preparedStatement.executeQuery(); /*\label{code:jdbc:executeQuery}*/
        DatabaseValue databaseValue = new DatabaseValue();
        databaseValue.setId(resultSet.getInt(1));
        databaseValue.setValue(resultSet.getString(2));
        return databaseValue; /*\label{code:jdbc:return}*/
      } catch (SQLException e) {
        // handle exception
      } finally {
        if (resultSet != null) {
          resultSet.close();
        }
      }
    } catch (SQLException e) {
      // handle exception
    } finally {
      if (preparedStatement != null) {
        preparedStatement.close();
      }
    }
  } catch (SQLException e) {
    // handle exception
  } finally {
    if (connection != null) {
      connection.close();
    }
  }
}
\end{lstlisting}


\textbf{TODO - popisovat na prikladoch?}

V štandardnej edícií javy sa objavuje aj balík \citet{java.sql}, ktorý poskytuje jednoduchšie API pre
konfiguráciu databáze. Rovnako pridáva funkcie pre distribuované tranzakcie a connection pooling.

\section{Spring JDBC Framework}

\citet{SpringJDBC} Framework je istou nadstavbou nad klasickým JDBC API, ktorého cieľom je uľahčiť jeho používanie.
Uľahčuje ošetrovanie výnimiek pri komunikácií s databázou, zjednodušuje možnosti používania tranzakcií,
znižuje množstvo opakujúceho sa kódu pri získavaní výsledkov z databázových volaní.

\textbf{TODO - popisovat na prikladoch?}

\section{MyBatis}

\citet{MyBatis}

\section{Hibernate}

\section{Spark}

\section{Kafka}


\chapter{Ďalšie možnosti pre vstup/výstup}

\section{Súbory, sockety, \dots}

