
\chapter{Úvod do statickej analýzy}

\section{Motivácia}

\section{MANTA}

\subsection{História}

História nástroja Manta Flow siaha do roku 2008, kedy ho vyvinula česká konzultačná spoločnosť Profinit s.r.o. ako interný nástroj.
Následne v roku 2013 sa autori projektu rozhodli založiť samostatnú spoločnosť a pokračovať v jeho vývoji.
V čele s Ing. Tomášom Krátkým založili spoločnosť Manta Tools s.r.o. známu ako MANTA.

Spoločnosť sa spolu s ČVUT zapojila do krantových programov ALFA a EPSILON Technologickej Agentury Českej Republiky (TAČR).
Vďaka nim získala v rokoch 2013 a 2017 dva granty v celkovej výške 1,49 milióna dolárov.

Spoločnosť sa presadila na zahraničnom trhu predovšetkým po výťazstve v súťaži ``Czech ICT Incubator @ Silicon Valley`` v roku 2014,
ktorú usporadúva Czech ICT Alliance. Po víťazstve bola založená prvá americká pobočka v San Francisku.

V súčastnosti pôsobí MANTA celosvetovo prostredníctvom vlastných pobočiek a siete regionálnych partnerov.
Medzi zákazníkov patrí napríklad spoločnosť Paypal, OBI, Vodafone, alebo Comcast.

O spoločnosti sa môžeme viac dočítať v článkoch od autorov \citet{MANTA} a \citet{MANTA_HISTORIA}.

\subsection{Manta Flow}

Manta Flow je nástrok umožňujúci automatickú analýzu programovacieho kódu (SQL, Java) a následný popis transformačnej logiky,
ktorý v ňom je obsiahnutý. Software je schopný rozpoznať aj ťažko čitateľné, na mieru napísané riadky programovacieho kódu.
Vďaka tejto vlastnosti dokáže v pomerne krátkom čase (obvykle niekoľko hodín) automaticky prečítať databáze o rozsahu
stotisícov i niekoľkých miliónov údajov a zostaviť z nich prehľadnú mapu datových tokov naprieč BI prostredím (Data Lineage).
To sa v praxi využíva najmä k optimalizácií datových skladov, znižovaniu nákladov na vývoj softwaru, vykonávanie dopadových
analýz a pri dokumentovaní prostredí pre potreby regulačných úradov.

