
\chapter{User Documentation for the \ToolName Tool \label{chapter:program}}

We created a set of plugins for Symbolic analysis library
for extracting data lineage information from data processing frameworks used in Java applications.
We also provide some test scenarios for frameworks we are handling.

We created the \ToolName tool, which is composition of the
Symbolic analysis library together with the plugins.

Here we describe installation and usage of the \ToolName tool.
We use standard Maven directory structure for source code.
In \Code{src/main} directory, application source and resource files
are located and in \Code{src/test} test sources and resources
are located.



\section{Software Requirements}

To successfully install the \ToolName tool, following is required:

\begin{itemize}
  \item Java JDK 1.8 installed
  \item Apache Maven installed and configured
    \begin{itemize}
      \item Recommended version is 3.3.9
    \end{itemize}
  \item Graphviz installed with tool \Code{dot} on PATH
    \begin{itemize}
      \item Recommended version is 2.40.1
    \end{itemize}
\end{itemize}

We recommend to use such versions of the software,
as these versions were used for testing and some compatibility
issues can be encountered using different versions.



\section{Installation}

To compile the \ToolName tool and create runnable \Code{JAR} file, use the command:
\begin{lstlisting}[language=shell]
        # mvn clean compile assembly:single
\end{lstlisting}

The command creates the executable \Code{target/resolver.jar}.
We recommend also to run all tests after instalation using:
\begin{lstlisting}[language=shell]
        # mvn test
\end{lstlisting}

When tests ends, the flow graph visualizations are saved in the \Code{target/img} directory.

As it could take a long time to run all tests (more than an hour), we provide TestNG
suites for each framework. This can be useful to verify that requested plugin is working
before trying to use it.

The test suite files are located in subdirectories of \Code{test/resources/tests}
and can be run using command:
\begin{lstlisting}[language=shell]
        # mvn test -DtestSuite=<suiteFile>
\end{lstlisting}


\section{Running the \ToolName Tool}

After instalation is completed, the \ToolName tool is ready to be used for computing
data lineage of target application using command line options:

\begin{lstlisting}[language=shell]
        # java -jar target/resolver.jar [OPTIONS]

        OPTIONS:
          [--entry <className> <methodName>]
          [--application-jar <fileName>]
          [--library-jar <fileName>]
          [--application-package <package>]
          [--output-directory <directoryName>]
          [--help]
\end{lstlisting}

\begin{itemize}
  \item \Code{-{}-entry} specify an entry point for the analysis.
    It can be some method, from which data lineage is computed.
    Fully qualified name of class should be provided and only methods
    without arguments and standard Java main method are supported.
  \item \Code{-{}-aplication-jar} specifies the application JAR file
    to be analysed by our \ToolName tool.
  \item \Code{-{}-library-jar} specifies the library JAR file
    to be known by our \ToolName tool.
    All library dependencies should be added for analysis.
  \item \Code{-{}-aplication-package} sets the root package of analysed application.
    This is needed, because of optimizations, when our \ToolName tool
    does not analyse classes that are outside of that package.
  \item \Code{-{}-output-directory} specifies where the result files are stored.
  \item \Code{-{}-help} display usage and exit.
\end{itemize}




\section{Result Graphs \label{chapter:program:graphs}}

The data lineage of an input program is represented by a flow graph,
where nodes are data sources and sinks and oriented edges are between
pair of nodes between which the data flows.

The graph can be visualized using Symbolic analysis library visualization tool
described in Section \ref{chapter:analysis:visualization}.

The \ToolName tool generates three types of files. First, the \Code{.dot} file contains graph definition
from which the \Code{dot} tool creates \Code{.pdf} and \Code{.svg} files with visualized result graph.

The result visualization was described in Section \ref{chapter:analysis:visualization}
for JDBC and I/O APIs.
Now we continue the list of node types from that section
that are relevant to Symbolic analysis library plugins:
\begin{itemize}
  \item \Code{FrameworkDataSource} - defines source of data in framework (like SQL query statement).
  \item \Code{FrameworkDataField} - defines resulting field from framework query.
  \item \Code{FrameworkAction} - defines the data sink (like SQL insert statement).
\end{itemize}

Plugins define also own attributes for identification of some valuable information
for data lineage.
Such attribute can be \Code{KAFKA\_TOPIC} that is used for identification of used
topic in Kafka Framework, or \Code{FILE\_NAME} that identifies a file name that was
used as some input in program, like the configuration file in MyBatis.
There are also much more attributes.




\section{Implementing Support for new Framework}

We implemented support for three data processing frameworks,
but in general, any framework can be handled.
It can be done by implementing a new plugin for Symbolic analysis library.

In Section \ref{chapter:implementation:interface} we described interface between
Symbolic analysis library and its plugins. Plugins should identify the data sources and sinks
and the library take care of the data flow between them.

Before the implementation of new plugin we strongly recomend to make familiar
with already implemented plugins and other classes used in that plugins:
\begin{itemize}
  \item \Code{JdbcTemplateAnalysisPlugin} and \Code{JdbcTemplateHandler} can be inspiration for the approach of callbacks.
  \item \Code{MyBatisAnalysisPlugin} can be inspiration, when plugin should work with external files or annotations.
    \begin{itemize}
      \item \Code{MyBatisAnnotationMapperSqlReader} - working with annotations
      \item \Code{MyBatisXmlMapperSqlReader} - working with external XML files
    \end{itemize}
\end{itemize}

